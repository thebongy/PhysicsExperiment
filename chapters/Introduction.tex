Simple Harmonic Motion appears in many regions of physics, as it serves for a useful mathematical model, to describe a variety of motions. Last year, we had chosen this topic and performed an experiment to observe the phenonmenon of beats, in a dual pendulum system, where one pendulum held a light source, and the other held a convex lens. The objective of the experiment was to obtain a sharp image of the light source on a screen, during the simultaneous oscilatory motion of both the pendulums. The success of this experiment, and its interesting observations, prompted us to re-take the SHM topic this year, and perform another experiment on the same.\\ \\
SHM describes the motion of a body to be sinusoidal in time, and this aspect of it often leads it to be used as an approximation of several motions in mechanics. One such example is the motion of a Simple Pendulum. The motion of a pendulum is assumed to follow an SHM for small amplitudes. The objective of this experiment is to determine how accurate this assumption is for bigger amplitudes. We do the same by predicting the angle of release of a body in pendulum motion, whose string has been cut during its oscillatory motion.

%%% Local Variables:
%%% mode: latex
%%% TeX-master: "../experiment"
%%% End: